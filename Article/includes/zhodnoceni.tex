\section{Zhodnocení}
Boid je velmi nenáročný na výpočetní výkon. Je možné jej paralelizovat, případně počítat na grafické kartě. 
\par
Výhoda modelu Boid spočívá v absenci detekce kolizí. Pokud mají agenti nastavenou správnou váhu, udržují si od sebe navzájem dostatečně velkou vzdálenost a není potřeba kolize počítat. 

\subsection{Problémy}
Přestože jednotliví agenti vykazují známky složitého chování, stále se řídí vesměs triviálními pravidly, které mohou zapříčinit nechtěné odchylky. Jednou z nich je tzv. semknutí dvou agentů, kteří se navzájem přetlačují a jeden druhému nedokáží ustoupit. To má za následek nechtěnné chování, kdy tito agenti většinou opustí hejno a může zapříčinit, že začnou procházet překážkami. 
\begin{figure}[H]
	\includegraphics[width=10cm]{people_problem_anim.png}
	\centering
	\caption{Srážka a semknutí dvou agentů v časech $t\in\{ 100, 115\} $}
\end{figure}
Jednotlivá pravidla modelu jsou nastavena tak, aby se model pokusil vyhýbat kolizím, ale nedokážou zajistit 100\% úspěšnost. Do jisté míry se tento problém dá oddálit správným nastavením vah. První testy chování Boid modelu dosahovaly přibližně 200 kolizí/s. Postupně se podařilo chování vyladit až na verzi bez žádné kolize. 
\par
Rozšíření o pravidlo vyhýbaní se překážkám, které Reynolds popsal ve své práci \cite{ReynoldsBoidNoBump} pracuje pouze se statickými objekty. Abhinav Golas, Rahul Narain a Ming Lin publikovali v roce 2014 článek Hybrid Long-Range Collision Avoidance for Crowd Simulation \cite{Golas2013}. Ve své práci popisují agenta, který umí predikovat pohyb jiných agentů a vyhnout se tak kolizi daleko dříve. 