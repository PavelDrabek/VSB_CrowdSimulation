\section{Úvod}
Umělá inteligence se v herním průmyslu využívá už od jeho vzniku. Ať už se jedná o propracovaný strategický systém, navigace po světě nebo detekce a přizpůsobení obtížnosti hráči. Oblast navigace by se dala rozdělit na navigaci samotného jedince a řízení více agentů. Tato práce se bude zabývat právě navigací více agentů. 
\par
Simulace davu má za cíl věrně napodobit chování velké množiny objektů. Zdánlivě složité chování však může být dosaženo definováním několika základních pravidel každého jedince. Chování samotného jedince se může jevit jako zmatené, avšak velká skupina stejně naprogramovaných jedinců může vykazovat známky inteligentního chování. 
\par
Takové chování uplatňujeme nejen ve filmovém a herním průmyslu, ale i v architektuře, nácviku vojenských strategií, navrhování evakuačních plánů, simulování požárních poplachů či chování robotů. \cite{gammaWeb} Díky moderním technologiím můžeme napodobit chování davu, které by nás stálo nejen velké množství času, ale i finančních prostředků. Například filmová bitva velkých armád, stadion plný fanoušků apod. Můžeme vytvořit i takové chování, se kterým bychom se v reálném životě nikdy nesetkali - ve filmu Drákula: Neznámá legenda (2014) ovládá hlavní postava desetitisíce netopýrů. 
\par 
Takovéto systémy většinou vykazují chaotické chování, protože nepatrná změna hodnoty jednoho parametru, byť jen zaokrouhlením, bude mít dopad na pozdější výsledek. Nicméně dá se zajistit i chování, které pro stejnou situaci chaotické chování nevykazují. 
\subsection{Cíl projektu}
Cílem projektu je vytvořit aplikaci, která bude využívat bioinspirované algoritmy pro umělou inteligenci. Aplikace bude simulovat chování vybraných objektů z reálného světa pomocí různých algoritmů. Cílem bude využít existující možnosti profesionálních herních enginů. 
\par
V projektu vytvoříme prostředí, kde bude možné jednotlivé chování simulovat a provedeme ve vytvořeném prostředí jednotlivé simulace. 
\par
Takto implementovaný algoritmus najde využití v oblasti virtuální reality, rozšířené reality, či herním průmyslu. 
